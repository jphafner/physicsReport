%%%%%%%%%%%%%%%%%%%%%%%%%%%%%%%%%%%%%%%%%%%%%%%%%%%%%%%%%%%%%%%%%%
% Sample template for MIT Junior Lab Student Written Summaries
% Available from http://web.mit.edu/8.13/www/Samplepaper/sample-paper.tex
% Last Updated April 12, 2007
%  Edited by Robert DeSerio August 2010.
%
% Adapted from the American Physical Societies REVTeK-4 Pages
% at http://publish.aps.org
%
% ADVICE TO STUDENTS: Each time you write a paper, start with this
%    template and save under a new filename.  If convenient, don't
%    erase unneeded lines, just comment them out.  Often, they
%    will be useful containers for information.
%
% Using pdflatex, images must be either PNG, GIF, JPEG or PDF.
%     Turn eps to pdf using epstopdf.
%%%%%%%%%%%%%%%%%%%%%%%%%%%%%%%%%%%%%%%%%%%%%%%%%%%%%%%%%%%%%%%%%%
% PREAMBLE

\documentclass[aps,twocolumn,secnumarabic,balancelastpage,amsmath,amssymb,nofootinbib,hyperref=pdftex]{revtex4}

% Documentclass Options
    % aps, prl, rmp stand for American Physical Society, Physical Review Letters, and Reviews of Modern Physics, respectively
    % twocolumn permits two columns, of course
    % nobalancelastpage doesn't attempt to equalize the lengths of the two columns on the last page
        % as might be desired in a journal where articles follow one another closely
    % amsmath and amssymb are necessary for the subequations environment among others
    % secnumarabic identifies sections by number to aid electronic review and commentary.
    % nofootinbib forces footnotes to occur on the page where they are first referenced
        % and not in the bibliography
    % REVTeX 4 is a set of macro packages designed to be used with LaTeX 2e.
        % REVTeX is well-suited for preparing manuscripts for submission to APS journals.


\usepackage{chapterbib}    % allows a bibliography for each chapter (each labguide has it's own)
\usepackage{color}         % produces boxes or entire pages with colored backgrounds
\usepackage{graphics}      % standard graphics specifications
\usepackage[pdftex]{graphicx}      % alternative graphics specifications
\usepackage{longtable}     % helps with long table options
\usepackage{epsf}          % old package handles encapsulated post script issues
\usepackage{bm}            % special 'bold-math' package
\usepackage{verbatim}			% for comment environment
\usepackage[colorlinks=true]{hyperref}  % this package should be added after all others
                                        % use as follows: \url{http://web.mit.edu/8.13}

%\addtolength\topmargin{-.5\topmargin} %cuts the top margin in half.

%
% And now, begin the document...
% Students should not have to alter anything above this line
%

\begin{document}
\title{Put Title Here}
\author         {Name}
\email          {email}
\homepage{Optional homepage}
\date{\today}
\affiliation{University of  West Florida Department of Physics}


\begin{abstract}
[Put abstract here]
\end{abstract}

\maketitle

%%%%%%%%%%%%%%%%%%%%%%%%%%%%%%%%%%%%%%%%%%%%%%%%%%%%%%%%%%%%%%%%%%

\section{Introduction}

[Put introduction here.]  We will cite one reference\cite{mayfield1998}
and use one equation.
\begin{equation}
e^{i\pi}-1 = 0
\label{eq:euler}
\end{equation}
What's not to love about Eq.~\ref{eq:euler}.

\section{Experiment}

\begin{figure}[htb]
\includegraphics[width=8cm]{sample-fig1}
\caption{[Put caption here.]\label{fig:samplefig}}
\end{figure}

[Put experiment section with figure here.]  Figure~\ref{fig:samplefig}
shows an experimental figure.  

\section{Results}

[Put results here.]  Table~\ref{tab:table1} shows a table.

\begin{table}[htb]
\caption{\label{tab:table1}[Put table caption here.]}
\begin{ruledtabular}
\begin{tabular}{cccccccc}
 &$r_c$ (\AA)&$r_0$ (\AA)&$\kappa r_0$&
 &$r_c$ (\AA) &$r_0$ (\AA)&$\kappa r_0$\\
\hline
Cu& 0.800 & 14.10 & 2.550 &Sn\footnotemark[1] & 0.680 & 1.870 & 3.700 \\
Ag& 0.990 & 15.90 & 2.710 &Pb\footnotemark[1] & 0.450 & 1.930 & 3.760 \\
Tl& 0.480 & 18.90 & 3.550 & & & & \\
\end{tabular}
\end{ruledtabular}
\footnotetext[1]{Here's the first, from Ref.~\cite{bevington2003}.}
\end{table}

\section{Conclusions}

[Put conclusions here.]

\bibliography{sample-paper}

[Don't forget you'll need to create a .bib file for your citations.

\appendix

\section{Comprehension Questions}

[Put answsers to comprehension questions here.]

\end{document}
