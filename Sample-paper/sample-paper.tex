%%%%%%%%%%%%%%%%%%%%%%%%%%%%%%%%%%%%%%%%%%%%%%%%%%%%%%%%%%%%%%%%%%
% Sample template for MIT Junior Lab Student Written Summaries
% Available from http://web.mit.edu/8.13/www/Samplepaper/sample-paper.tex
%
% Last Updated April 12, 2007
%
% Adapted from the American Physical Societies REVTeK-4 Pages
% at http://publish.aps.org
%
% ADVICE TO STUDENTS: Each time you write a paper, start with this
%    template and save under a new filename.  If convenient, don't
%    erase unneeded lines, just comment them out.  Often, they
%    will be useful containers for information.
%
% Using pdflatex, images must be either PNG, GIF, JPEG or PDF.
%     Turn eps to pdf using epstopdf.
%%%%%%%%%%%%%%%%%%%%%%%%%%%%%%%%%%%%%%%%%%%%%%%%%%%%%%%%%%%%%%%%%%


%%%%%%%%%%%%%%%%%%%%%%%%%%%%%%%%%%%%%%%%%%%%%%%%%%%%%%%%%%%%%%%%%%
% PREAMBLE
% The preamble of a LaTeX document is the set of commands that precede
% the \begin{document} line.  It contains a \documentclass line
% to load the REVTeK-4 macro definitions and various \usepackage
% lines to load other macro packages.
%
% ADVICE TO STUDENTS: This preamble contains a suggested set of
%     class options to generate a ``Junior Lab'' look and feel that
%     facilitate quick review and feedback from one's peers, TA's
%     and section instructors.  Don't make substantial changes without
%     first consulting your section instructor.
%%%%%%%%%%%%%%%%%%%%%%%%%%%%%%%%%%%%%%%%%%%%%%%%%%%%%%%%%%%%%%%%%%

\documentclass[aps,twocolumn,twoside,secnumarabic,balancelastpage,amsmath,amssymb,nofootinbib,hyperref=pdftex]{revtex4}

% Documentclass Options
    % aps, prl, rmp stand for American Physical Society, Physical Review Letters, and Reviews of Modern Physics, respectively
    % twocolumn permits two columns, of course
    % nobalancelastpage doesn't attempt to equalize the lengths of the two columns on the last page
        % as might be desired in a journal where articles follow one another closely
    % amsmath and amssymb are necessary for the subequations environment among others
    % secnumarabic identifies sections by number to aid electronic review and commentary.
    % nofootinbib forces footnotes to occur on the page where they are first referenced
        % and not in the bibliography
    % REVTeX 4 is a set of macro packages designed to be used with LaTeX 2e.
        % REVTeX is well-suited for preparing manuscripts for submission to APS journals.


%\usepackage{lgrind}        % convert program listings to a form includable in a LaTeX document
\usepackage{chapterbib}    % allows a bibliography for each chapter (each labguide has it's own)
\usepackage{color}         % produces boxes or entire pages with colored backgrounds
\usepackage{graphics}      % standard graphics specifications
\usepackage[pdftex]{graphicx}      % alternative graphics specifications
\usepackage{longtable}     % helps with long table options
\usepackage{epsf}          % old package handles encapsulated post script issues
\usepackage{bm}            % special 'bold-math' package
\usepackage{verbatim}			% for comment environment
%\usepackage{asymptote}     % For typesetting of mathematical illustrations
%\usepackage{thumbpdf}
\usepackage[colorlinks=true]{hyperref}  % this package should be added after all others
                                        % use as follows: \url{http://web.mit.edu/8.13}

%Next line moves text up a bit without changing text area size  
\addtolength\topmargin{-.5\topmargin} %increases the top margin by half.

%
% And now, begin the document...
% Students should not have to alter anything above this line
%

\begin{document}
\title{Writing Scientific Reports Using \LaTeX\footnote{
This document was adapted from the original---written for MIT Junior lab students,
\url{web.mit.edu/8.13/www/}
}}
\author         {Aaron Wade}
\email          {awade1@uwf.edu}
\homepage{http://uwf.edu/awade1/ModernLab/}
\date{\today}
\affiliation{University of West Florida Physics Department}


\begin{abstract}
This part, the abstract is an essential part of a scientific paper---often the only
component freely viewable from search engines.
It should briefly summarize the background, the purpose,
the method, and most importantly,
the quantitative results with errors.
Based on those, a conclusion may be drawn.
In this paper, we present a \LaTeX\ template for formal reports in PHY4803L.
It is based on the REV\TeX\ document class from the
American Physical Society---a standard for the
Physical Review journals as well as many others.
Your paper should demonstrate your mastery of the entire experiment.
It should be neat in appearance with correct English.
It should be concise; four single-spaced pages including figures should suffice.
Not included in the four-page limit, appendices can be used to
present data that is summarized in the main body,
for derivations referred to in the main body,
and for answers to questions posed in the experimental guides.

\end{abstract}

\maketitle

%%%%%%%%%%%%%%%%%%%%%%%%%%%%%%%%%%%%%%%%%%%%%%%%%%%%%%%%%%%%%%%%%%

\section{Introduction and Theory}

The introduction typically places
the current work in context with prior work
and explains what new physics is involved
and why the article is worth reading.
For your articles, use one or more paragraphs to
succinctly explain the motivation, purpose and relevant
background to the experiment.
This should be done at a level so that another
lab student could follow your development.

A full theory section should not normally be needed for our advanced lab experiments.
So use the introduction to present the main physics variables and formulas you will use.
Trace their origin to the physics involved.
Don't provide derivations, but do describe
what new assumptions are needed.
Formulas involving measurement conversions, instrument settings
or other apparatus details should be relegated to the apparatus and experiment section.

Here we will use the introduction to discuss technical writing issues. 

One resource for developing into a strong
technical writer is the UF Reading and Writing Center\cite{rwc}.
Students can even receive free consultations 
on their written reports through this office!

An important part of your education as a physicist is learning to
use standard tools for sharing your work with others.
In Advanced Lab, we will instruct you in the use of \LaTeX\ for
writing scientific papers in a widely accepted professional style.
Articles submitted for publication in a professional journal
must be suitably formatted according to the journal guidelines.
Physical Review Letters and many others adhere to the
\href{http://forms.aps.org/author/styleguide.pdf}{APS Physics Review Style and Notation Guide}.
The \href {https://authors.aps.org/revtex4/}{REV\TeX\ homepage}
has additional reference material.

The source files\footnote
{\url{http://uwf.edu/awade1/ModernLab/LabReport/sample-paper.zip}}
for this document may be used as a template for your Advanced Lab papers.
Spending a few hours studying and altering the sample-paper.tex
and sample-paper.bib files will allow you to develop sufficient mastery of \LaTeX\ to easily
generate all manner of technical documents.  Specific instructions
for compiling \LaTeX\ documents on Windows operating systems are
contained in the appendices.

The writing process involves at least three distinct steps: prewriting or outlining,
drafting, and revising or editing.  Given the tight time constraints in Advanced
Lab, students are advised to begin the drafting process \textbf{before}
finishing their lab sessions.  Most of the first draft can be accomplished during
the latter sessions of an experiment.

%%%%%%%%%%%%%%%%%%%%%%%%%%%%%%%%%%%%%%%%%%%%%%%%%%%%%%%%%%%%%%%%%%%%%%%%%%%%
% This is a basic figure drawn using the asymptote package
% see http://asymptote.sourceforge.net/ for more information
%\begin{figure}
%\centering
%\begin{asy}
%size(3cm);
%draw(unitcircle);
%\end{asy}
%\caption{Embedded Asymptote Figure} \label {fig:asymptote1}
%\end{figure}
%%%%%%%%%%%%%%%%%%%%%%%%%%%%%%%%%%%%%%%%%%%%%%%%%%%%%%%%%%%%%%%%%%%%%%%%%%%%%
%\subsection{Expository Writing}
The essence of expository writing is the communication of understanding
through a clear and concise presentation of predominately factual
material.\cite{mayfield1998,pritchard1990} Most people cannot
compose successful expository prose unless they put the need to
communicate foremost among their priorities. Two things predominate
in generating understanding in the reader:
\begin{description}
\item[Organization:] The reader must be provided with an overview or
outline, know how each fact fits into that overall picture,
and must be specifically alerted about any especially important fact.
Furthermore, the facts must be presented in a logical order---so
that fact 17 is not important for understanding fact 12.
\item[Depth:] Bearing in mind the preexisting
knowledge of the reader, the writer must budget the length of
discussion allotted to each topic in proportion to its importance.
\end{description}

Writing a journal-like article for the lab report is great practice
for improving your technical writing.
Thus we urge you to concentrate on your overall presentation,
not only on the facts themselves.
We strongly recommend that you:
\begin{enumerate}
\item Base your report on an outline.
\item Begin each paragraph with a topic sentence which expresses the
main area of concern and the main conclusion of the paragraph. Put
less important material later in the paragraph.
\end{enumerate}

Point 2 is frequently absent in novice reports; topic sentences are your
mechanism for telling the reader what is under discussion
and where it fits into the overall picture.
You can check your topic sentences by reading them in order, i.e.,
omit all the following sentences in each paragraph; this should
give a fair synopsis of your paper.

If you are writing up results you obtained with a partner,
use we for work performed together and I for work performed alone.
Use the past tense for your procedure and analysis,
and the present for your results. 
``LiF xray diffraction angles \emph{were} measured to
$\pm 0.2^\circ$ and \emph{are} consistent with an FCC lattice
with a spacing $a_0=(4.035\pm0.014)\,\mbox{nm}$.''
Note that units are in normal (not math) fonts; the source file
shows how to make this happen while inside the \LaTeX\ math mode.

\textbf{Lastly: Remember to proofread your paper for spelling and grammar
 mistakes.  Few things are as offensive to a reviewer as careless
 writing and such mistakes will count against you!}

\section{Apparatus and Experiment}

With reference to one or more figures, this section describes the apparatus
and procedures that give rise to the raw data.  Also discuss the data's 
random errors and the sources and sizes of possible systematic errors.
Include here critical observations of any noteworthy issues associated with the apparatus.

The apparatus figure should contain a block diagram or schematic of the equipment and perhaps
include the most important signal processing steps.
The figure should be referenced and placed as early as possible in this section.

Place additional information within the figures or in their captions
to help you stay within the four page limit.
\textbf{Example first sentence of an experimental section:}
The experimental apparatus consists of a specially prepared chemical
sample containing $^{13}$CHCl$_3$, an NMR spectrometer, and a control
computer, as shown in Fig.~\ref{fig:samplefig}.

%
% Note, when including figures in a TeX document,
% we suggest you first create the figures as
% PDF files and then include them in the following way -
% without the .pdf suffix)
%

\begin{figure}[htb]
\includegraphics[width=8cm]{sample-fig1}
\caption{Figures should be inserted into the text in their natural positions.
%They must be JPG or PNG bitmap figures or PDF images.
Command options can be used to crop, scale, or rotate the figure.
The size of this graphic was set by the width command.
The aspect ratio defaults to 1.0 if the height is not also set.
When creating figures, choose large font sizes in graph labels and other figure information;
the figure should be legible when scaled to fit in a single column.
This part---the caption---should be clear and comprehensible.
Use the caption to elaborate on specific issues, features, complications, or operating procedures.
Adapted from \cite{melissinos1966,melissinos2003}.\label{fig:samplefig}}
\end{figure}

\section{Analysis and Results}

This section should demonstrate how the raw data lead to the main results.
Make a complete estimate of the uncertainties in your results---both
random and systematic.

In some cases, it is proper to put data and analysis in the experiment section,
particularly if it is more about the apparatus and its parameters than about the main physics
of the experiment.
And in some cases, it is proper to describe procedures in this section.
When it aids the logical flow of the paper, keep procedures,
data, analysis, and results together.

Either here or in the previous section, be sure to display representative raw data.
Where there is an abundance of data, consider using an appendix to present it.
See, for example, Fig.~\ref{fig:landscapegraphic}.

\begin{figure}[htb]
\includegraphics[width=9cm]{sample-fig2.pdf}	%trim = 0in 0.5in 0in 0.5in, (would shave off the top and bottom of the figure)
\caption{Sample figure describing a set of data, fit procedures and
results. Use the caption space to provide more details about the
fitting procedure, results or implications if you do not have
sufficient room in the main body of text.
%This figure was created using the Matlab script at
% \url{web.mit.edu/8.13/matlab/fittemplate07.m}
\label{fig:calibration}
}
\end{figure}


\begin{figure}[htb]
\includegraphics[width=9cm]{sample-fig3.pdf}			%trim = 0in 0.5in 0in 0.5in, (would shave off the top and bottom of the figure)
\caption{Sample figure showing overall physical relationship you set
out to test.
%This figure was created using the Matlab script at
%\url{web.mit.edu/8.13/matlab/fittemplate07.m}
\label{fig:frenchtaylor}
}
\end{figure}

Often the raw data are analyzed in a specific way that needs to be clearly
communicated to the reader.  For example, the peak positions in a
spectrum may be required.  A graphic demonstrating a typical
fit result, functional model, and reduced $\chi^2$ is shown in
Fig.~\ref{fig:calibration}. Finally, there should be a graph or table
which summarizes the experimental data, and which conveys the primary
findings of the laboratory exercise.
For example, the Geiger-Nuttall relationship as shown in Fig.~\ref{fig:frenchtaylor}
or Table~\ref{tab:table1} containing results of xray spectra analyses.

\begin{table}[htb]
\caption{\label{tab:table1}An example table with footnotes.  Note
that several entries share the same footnote. Always use a preceding
zero in the data you record in tables.  \textbf{Always display units}.
Inspect the \LaTeX\ input for this table to see exactly how it is
done.}
\begin{ruledtabular}
\begin{tabular}{cccccccc}
 &$r_c$ (\AA)&$r_0$ (\AA)&$\kappa r_0$&
 &$r_c$ (\AA) &$r_0$ (\AA)&$\kappa r_0$\\
\hline
Cu& 0.800 & 14.10 & 2.550 &Sn\footnotemark[1] & 0.680 & 1.870 & 3.700 \\
Ag& 0.990 & 15.90 & 2.710 &Pb\footnotemark[1] & 0.450 & 1.930 & 3.760 \\
Tl& 0.480 & 18.90 & 3.550 & & & & \\
\end{tabular}
\end{ruledtabular}
\footnotetext[1]{Here's the first, from Ref.~\cite{bevington2003}.}
\end{table}

Additional graphics, such as Figure~\ref{fig:calibration}, should be well
thought out and crafted to maximize their information content while
retaining clarity of expression.
Try to avoid the temptation to inundate the reader with too many
graphics.  It is worth spending some time thinking of how best to
present information rather than just creating graph after graph of
uninformative data.

% Use the comment environment rather than leading % signs to
% comment out a section of the file as follows.
% This environment is defined in the verbatim macro package.
 
\begin{comment}

%Everything between the begin and end comment command is not processed.
%This is a good way to temporarily delete a section of material you might
%decide later to put back in.

\begin{figure*}[htb]
\includegraphics[angle=0,width=10cm]{sample-fig6}
\caption{Sample paneled figure created in Matlab using the
subplot(2,2,x) command where x is the element of the plot array into
which all subsequent commands such as plot(x,y) and xlabel('Volts'), etc., get processed.
Use the caption space to provide more details
about the data, their acquisition or how they were processed if you do
not have sufficient room in the main body of text.  Figures can be
rotated using the angle option, see the TeX file for details.  If a
figure is to be placed after the main text use the ``figure*'' option
to make it extend over two columns, see the \LaTeX file for how this
was done.}
\label{fig:panel2x2}
\end{figure*}

\end{comment}


\section{Conclusions}

Summarize and discuss the findings of the experiment.
Report all your results with appropriate significant digits, units, and uncertainties,
e.g., $Q = (2.12 \pm 0.06)\times 10^{10}$~disintegrations s$^{-1}$. 
When appropriate, compare your results with
theoretical expectations or other experimental values
with respect to the standard deviations of the quantities involved.
Make suggestions for improvements and describe additional experiments
that could be attempted with this or an improved apparatus.
Be adventurous with your suggestions.

It is worth mentioning here some thoughts on \textbf{ethics and writing
in science}.

When you read the report of a physics experiment in a reputable
journal (e.g., Physical Review Letters) you can generally assume it
represents an honest effort by the authors to describe exactly what
they observed. You may doubt the interpretation or the theory they
create to explain the results. But at least you trust that if you
repeat the manipulations as described, you will get essentially the
same experimental results.

Nature is the ultimate enforcer of truth in science. If subsequent
work proves a published measurement is wrong by substantially more
than the estimated error limits, a reputation shrinks. If fraud is
discovered, a career may be ruined. So most professional scientists
are very careful about the records they maintain and the results and
errors they publish.

In keeping with the spirit of trust in science, Advanced Lab instructors
will assume that what you record in your lab book and report in your
written and oral presentations is exactly what you have observed.

Using other people's words without acknowledgement is a
serious intellectual crime and possible causes for dismissal from the University.
The appropriate way to incorporate an
idea which you have learned from a textbook or other reference is to
study the point until you understand it and then put the text aside
and state the idea in your own words.
Fabrication or falsification of data and using the results of
another person's work without acknowledgement are offenses of similar gravity.

One often sees, in a scientific journal, phrases such as ``Following
Bevington and Melissinos \cite{bevington2003, melissinos1966} ...''
This means that the author is following the ideas or logic of these
authors and not their exact words.
If you do choose to quote material, it is not sufficient just to
include the original source among the list of references at the end of
your paper. If a few sentences or more are imported from another
source, that section should be

\begin{quote}indented on both sides or enclosed in
quotes, and attribution must be given immediately in the form of a
reference note.\cite{melissinos1966}
\end{quote}

If you have any question at all about attribution of sources, please
see the course instructor.
The University has produced an
\href{http://uwf.edu/library/research_help/using-sources-ethically/}{online video}
with additional information about how to avoid plagiarism.

\begin{figure*}[htb] %The {figure*} environment is used for two-column figures.
\label{fig:crop}
\includegraphics[trim=0.2cm 0.2cm 3in 8.5in, clip=true, width=6in]{sample-fig4}
%The trim option in the includegraphics command trims off the
%left, bottom, right, and top lengths from the PDF file containing the figure.
\caption{
This is a two-column figure using the \texttt{figure*} environment. 
Two column figures can't be on the first page and \LaTeX\ often
has trouble with their placement.
}
\begin{comment}
\end{comment}
\end{figure*}

\section{Bibliography Remarks}

Bibliographies are very important in Advanced Lab papers.  Beyond the
requisite citation of source material, they provide evidence of your
literature research beyond the experimental guides.
Literature searches, appropriate references to other research, and bibliographies
are an integral part of experimental research.
Bibliographic entries are made within a separate `.bib'
file which gets attached during the process of building a final PDF
document.  See the bibliography file sample-paper.bib (included in the zip file)
for details on several types of bibliographic entries and their required and
optional fields.

%%%%%%%%%%%%%%%%%%%%%%%%%%%%%%%%%%%%%%%%%%%%%%%%%%%%%%%%%%%%%%%%%%%%%%%%%%%%%
% Place all of the references you used to write this paper in a file
% with the file name then referenced in the \bibliography command as follows
%%%%%%%%%%%%%%%%%%%%%%%%%%%%%%%%%%%%%%%%%%%%%%%%%%%%%%%%%%%%%%%%%%%%%%%%%%%%%

\begin{comment}

\section{Sectioning}

The recommended \verb+aps+ option in the \verb+documentclass+ command
creates the sectioning style shown here. 

\subsection{Subsection Title}

In a short paper, you shouldn't get below subsections.

\subsubsection{Subsubsection Title}

Subsubsections look like this.

\end{comment}

\section{Typesetting Mathematics}

One of the great powers of \LaTeX\ is it's ability to typeset all
manner of mathematical expressions.  While it does take a short
while to get used to the syntax, it will soon become second nature.
Numbered, single-line equations are the most common and are usually
referenced in the text; e.g., see Eq.~\ref{eq:first-equation}.
%
\begin{equation}
   \chi_+(p)\alt{\bf [}2|{\bf p}|(|{\bf p}|+p_z){\bf ]}^{-1/2}
   \left(
   \begin{array}{c}
      |{\bf p}|+p_z\\
      px+ip_y
   \end{array}\right)
\,. \label{eq:first-equation}
\end{equation}
%
% Be sure there is NO EMPTY LINE after \end{quation} and before the
% following lines, if you do not want a new paragraph to start there
% (and be indented).
%
Occasionally, long equations which span more than one line of a two-column page
may be required.  A good solution is to split-up
the equation into multiple lines and label all with a single
equation number, like in Equation~\ref{eq:multilineeq}.  See the
\LaTeX\ file to see how this is done.

\begin{eqnarray}
  \sum \vert M^{\text{viol}}_g \vert ^2
   &=&  g^{2n-4}_S(Q^2)~N^{n-2} (N^2-1)
\nonumber
\\
   &&   \times \left( \sum_{i<j}\right) \sum_{\text{perm}}
            \frac{1}{S_{12}}  \frac{1}{S_{12}} \sum_\tau c^f_\tau
\,.
\label{eq:multilineeq}
\end{eqnarray}

It is often useful to group related equations to denote their
relationship, e.g., in a derivation.  Enclosing single-line and
multiline equations in \verb+\begin{subequations}+ and
\verb+\end{subequations}+ will produce a set of equations that are
``numbered'' with letters, as shown in Equations.~(\ref{subeq:1}) and
(\ref{subeq:2}) below:
\begin{subequations}
\label{eq:whole}
\begin{equation}
  \left\{
      \text{abc123}456abcdef\alpha\beta\gamma\delta1234556\alpha\beta
       \frac{1\sum^{a}_{b}}{A^2}
  \right\}
%
\,\label{subeq:1}
\end{equation}
\begin{eqnarray}
  {\cal M} &=& ig_Z^2(4E_1E_2)^{1/2}(l_i^2)^{-1}
                (g_{\sigma_2}^e)^2\chi_{-\sigma_2}(p_2).\label{subeq:2}
\end{eqnarray}
\end{subequations}
Note how you can also create a reference to Eqs.~\ref{eq:whole}, i.e., all subequations, by proper
location of the \verb+\label+ command that creates the references.

Mathematics can also be placed directly in the text using
delimiters: $\vec{\psi_1} = |\psi_1\rangle \equiv c_0|0\rangle +
c_1|1\rangle \chi^2 \approx
\prod\sum\left[\frac{y_i-f(x_i)}{\sigma_i}\right]^2 |\psi_1\rangle
\sim \lim_{\mu \rightarrow \infty}p(x;\mu) \geq \frac{1}{\sqrt{2 \pi
\mu}} e^{-(x-\mu)^2 / 2\mu}P(x) \ll \int_{-\infty}^x p(x')dx'a
\times b \pm c \Rightarrow \nabla \hbar$.


%The next line builds an inserts the bibliograhy.

\bibliography{sample-paper}

%%%%%%%%%%%%%%%%%%%%%%%%%%%%%%%%%%%%%%%%%%%%%%%%%%%%%%%%%%%%%%%%%%%%%%%%%%%%%

\appendix
\section{\LaTeX\ Under Windows}
MiK\TeX\ (pronounced \emph{mik-tech} is a freely available, implementation of
\TeX\ and related programs available from \url{www.miktex.org}. Note
that MiK\TeX\ itself runs from a command line prompt and is not terribly
convenient.  A very nice \LaTeX\ editor/shell called \TeX nicCenter
is also freely available from
\url{www.texniccenter.org}. 

Once you've installed the above software, you will need to download
\href{uwf.edu/awade1/ModernLab/LabReport/sample-paper.zip}{sample-paper.zip}
and extract files listed below to their
own directory in order to `rebuild' this document from scratch.
You should probably build your report starting from
bare-paper.tex which will lessen the amount of material
that needs to be deleted.


\begin{description}
\item[sample-paper.tex]\ \ The main paper.
\item[sample-paper.bib]\ \ The text file where reference information is located.
\item[figure files]\ \ The files sample-fig1.pdf through sample-fig4.pdf files are PDF-viewable
figures requested in the sample-paper.tex file.  They can be created with the PDFCreator printer available on the lab computers.
\item[sample-paper.pdf]\ \ This is the PDF file created by the \LaTeX\ compiler.
\item[bare-paper.tex]\ \ A nearly bare template.
\end{description}
Other output files such as a sample-paper.log file are also made during compilation.

Use the \verb+LaTeX=>PDF+ Output Profile in the \TeX nicCenter toolbar to directly create pdf files
from \LaTeX\ source files.  The toolbar includes `Build current file,' `View Output,' and
`Build and view current file' options.  The first and last recompile the source
into a pdf file.  The middle simply views the most current PDF\@.
Typically two or three repeated calls to build the PDF output file are necessary to resolve any nested
references.

The \verb+\bibliography{sample-paper}+ command
generates the bibliography at that point in the document.
It invokes the `bibtex' macro package that reads in the
bibliography file `sample-paper.bib' allowing citation references to
be resolved.
Additional files get
regenerated when you build your PDF document.

\textbf{\TeX nicCenter, includes a spell checker that ignores most \LaTeX\ 
commands.  Be sure that you spell check and check grammar before handing in your paper!}

\section{PDFCreator and \LaTeX}

Excel and MatLab are the most common analysis tools used by Advanced Lab students
for data analysis and graphing.  LabVIEW is our most commonly used
software for experimental control and display.
All lab software printable output can be saved directly into a `PDF'
format by printing to the PDFCreator printer driver.
If the output is a full $8.5\times 11$ inch PDF page, and you only 
want an area, it can be cropped when inserted into the \LaTeX\ file.
See the source file for the syntax.  This has been done on Fig.~\ref{fig:crop}.
This figure uses the two-column \verb+figure*+ environment for a figure that
needs the extra width.

%%%%%%%%%%%%%%%%%%%%%%%%%%%%%%%%%%%%%%%%%%%%%%%%%%%%%%%%%%%%%%%%%%%%%%%%%%%%%
%%%%%%%%%%%%%%%%%%%%%%%%%%%%%%%%%%%%%%%%%%%%%%%%%%%%%%%%%%%%%%%%%%%%%%%%%%%%%


%%%%%%%%%%%%%%%%%%%%%%%%%%%%%%%%%%%%%%%%%%%%%%%%%%%%%%%%%%%%%%%%%%%%%%%%%%%%%

% Surround figure environment with turnpage environment for landscape presentation
\begin{turnpage}
\begin{figure*}[p]
\includegraphics[width=20cm]{sample-fig5}
\caption{For very large plots where important detail might be lost
if too compressed, it can be convenient to use the `turnpage'
environment for displaying in landscape mode. e.g., any experiment
where a data set is acquired at several angular positions (21cm,
e/m, Rutherford) or is time varying (Physics of Alpha Decay and
Pulsed NMR.)  These full page graphics are usually best kept in
appendices so as not to impede the flow of the paper.  Note that
large tables can also be presented in this landscape environment if
desired \label{fig:landscapegraphic}}
\end{figure*}
\end{turnpage}

% tables should appear as floats within the text
%
% Here is an example of the general form of a table:
% Fill in the caption in the braces of the \caption{} command. Put the label
% that you will use with \ref{} command in the braces of the \label{} command.
% Insert the column specifiers (l, r, c, d, etc.) in the empty braces of the
% \begin{tabular}{} command.
% The ruledtabular enviroment adds doubled rules to table and sets a
% reasonable default table settings.
% Use the table* environment to get a full-width table in two-column
% Add \usepackage{longtable} and the longtable (or longtable*}
% environment for nicely formatted long tables. Or use the the [H]
% placement option to break a long table (with less control than
% in longtable).
% \begin{table}%[H] add [H] placement to break table across pages
% \caption{\label{}}
% \begin{ruledtabular}
% \begin{tabular}{}
% Lines of table here ending with \\
% \end{tabular}
% \end{ruledtabular}
% \end{table}

% To convert program (e.g., C++ Fortran, Matlab, LaTeX\) listings to a
% form easily includable in a \LaTeX\ document
%
% type lgrind -s to see options
% lgrind -llatex -i sample-paper.tex > sampleinputtex
% creates a file sampleinput.tex which can then be included into this
% document simply by uncommenting the next line
%\lgrindfile{testinput.tex}

\end{document}
